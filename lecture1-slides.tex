\PassOptionsToPackage{unicode=true}{hyperref} % options for packages loaded elsewhere
\PassOptionsToPackage{hyphens}{url}
%
\documentclass[ignorenonframetext,]{beamer}
\usepackage{pgfpages}
\setbeamertemplate{caption}[numbered]
\setbeamertemplate{caption label separator}{: }
\setbeamercolor{caption name}{fg=normal text.fg}
\beamertemplatenavigationsymbolsempty
% Prevent slide breaks in the middle of a paragraph:
\widowpenalties 1 10000
\raggedbottom
\setbeamertemplate{part page}{
\centering
\begin{beamercolorbox}[sep=16pt,center]{part title}
  \usebeamerfont{part title}\insertpart\par
\end{beamercolorbox}
}
\setbeamertemplate{section page}{
\centering
\begin{beamercolorbox}[sep=12pt,center]{part title}
  \usebeamerfont{section title}\insertsection\par
\end{beamercolorbox}
}
\setbeamertemplate{subsection page}{
\centering
\begin{beamercolorbox}[sep=8pt,center]{part title}
  \usebeamerfont{subsection title}\insertsubsection\par
\end{beamercolorbox}
}
\AtBeginPart{
  \frame{\partpage}
}
\AtBeginSection{
  \ifbibliography
  \else
    \frame{\sectionpage}
  \fi
}
\AtBeginSubsection{
  \frame{\subsectionpage}
}
\usepackage{lmodern}
\usepackage{amssymb,amsmath}
\usepackage{ifxetex,ifluatex}
\usepackage{fixltx2e} % provides \textsubscript
\ifnum 0\ifxetex 1\fi\ifluatex 1\fi=0 % if pdftex
  \usepackage[T1]{fontenc}
  \usepackage[utf8]{inputenc}
  \usepackage{textcomp} % provides euro and other symbols
\else % if luatex or xelatex
  \usepackage{unicode-math}
  \defaultfontfeatures{Ligatures=TeX,Scale=MatchLowercase}
\fi
\usetheme[]{metropolis}
% use upquote if available, for straight quotes in verbatim environments
\IfFileExists{upquote.sty}{\usepackage{upquote}}{}
% use microtype if available
\IfFileExists{microtype.sty}{%
\usepackage[]{microtype}
\UseMicrotypeSet[protrusion]{basicmath} % disable protrusion for tt fonts
}{}
\IfFileExists{parskip.sty}{%
\usepackage{parskip}
}{% else
\setlength{\parindent}{0pt}
\setlength{\parskip}{6pt plus 2pt minus 1pt}
}
\usepackage{hyperref}
\hypersetup{
            pdftitle={Coding 1: Lecture 1},
            pdfborder={0 0 0},
            breaklinks=true}
\urlstyle{same}  % don't use monospace font for urls
\newif\ifbibliography
\usepackage{color}
\usepackage{fancyvrb}
\newcommand{\VerbBar}{|}
\newcommand{\VERB}{\Verb[commandchars=\\\{\}]}
\DefineVerbatimEnvironment{Highlighting}{Verbatim}{commandchars=\\\{\}}
% Add ',fontsize=\small' for more characters per line
\usepackage{framed}
\definecolor{shadecolor}{RGB}{248,248,248}
\newenvironment{Shaded}{\begin{snugshade}}{\end{snugshade}}
\newcommand{\AlertTok}[1]{\textcolor[rgb]{0.94,0.16,0.16}{#1}}
\newcommand{\AnnotationTok}[1]{\textcolor[rgb]{0.56,0.35,0.01}{\textbf{\textit{#1}}}}
\newcommand{\AttributeTok}[1]{\textcolor[rgb]{0.77,0.63,0.00}{#1}}
\newcommand{\BaseNTok}[1]{\textcolor[rgb]{0.00,0.00,0.81}{#1}}
\newcommand{\BuiltInTok}[1]{#1}
\newcommand{\CharTok}[1]{\textcolor[rgb]{0.31,0.60,0.02}{#1}}
\newcommand{\CommentTok}[1]{\textcolor[rgb]{0.56,0.35,0.01}{\textit{#1}}}
\newcommand{\CommentVarTok}[1]{\textcolor[rgb]{0.56,0.35,0.01}{\textbf{\textit{#1}}}}
\newcommand{\ConstantTok}[1]{\textcolor[rgb]{0.00,0.00,0.00}{#1}}
\newcommand{\ControlFlowTok}[1]{\textcolor[rgb]{0.13,0.29,0.53}{\textbf{#1}}}
\newcommand{\DataTypeTok}[1]{\textcolor[rgb]{0.13,0.29,0.53}{#1}}
\newcommand{\DecValTok}[1]{\textcolor[rgb]{0.00,0.00,0.81}{#1}}
\newcommand{\DocumentationTok}[1]{\textcolor[rgb]{0.56,0.35,0.01}{\textbf{\textit{#1}}}}
\newcommand{\ErrorTok}[1]{\textcolor[rgb]{0.64,0.00,0.00}{\textbf{#1}}}
\newcommand{\ExtensionTok}[1]{#1}
\newcommand{\FloatTok}[1]{\textcolor[rgb]{0.00,0.00,0.81}{#1}}
\newcommand{\FunctionTok}[1]{\textcolor[rgb]{0.00,0.00,0.00}{#1}}
\newcommand{\ImportTok}[1]{#1}
\newcommand{\InformationTok}[1]{\textcolor[rgb]{0.56,0.35,0.01}{\textbf{\textit{#1}}}}
\newcommand{\KeywordTok}[1]{\textcolor[rgb]{0.13,0.29,0.53}{\textbf{#1}}}
\newcommand{\NormalTok}[1]{#1}
\newcommand{\OperatorTok}[1]{\textcolor[rgb]{0.81,0.36,0.00}{\textbf{#1}}}
\newcommand{\OtherTok}[1]{\textcolor[rgb]{0.56,0.35,0.01}{#1}}
\newcommand{\PreprocessorTok}[1]{\textcolor[rgb]{0.56,0.35,0.01}{\textit{#1}}}
\newcommand{\RegionMarkerTok}[1]{#1}
\newcommand{\SpecialCharTok}[1]{\textcolor[rgb]{0.00,0.00,0.00}{#1}}
\newcommand{\SpecialStringTok}[1]{\textcolor[rgb]{0.31,0.60,0.02}{#1}}
\newcommand{\StringTok}[1]{\textcolor[rgb]{0.31,0.60,0.02}{#1}}
\newcommand{\VariableTok}[1]{\textcolor[rgb]{0.00,0.00,0.00}{#1}}
\newcommand{\VerbatimStringTok}[1]{\textcolor[rgb]{0.31,0.60,0.02}{#1}}
\newcommand{\WarningTok}[1]{\textcolor[rgb]{0.56,0.35,0.01}{\textbf{\textit{#1}}}}
\setlength{\emergencystretch}{3em}  % prevent overfull lines
\providecommand{\tightlist}{%
  \setlength{\itemsep}{0pt}\setlength{\parskip}{0pt}}
\setcounter{secnumdepth}{0}

% set default figure placement to htbp
\makeatletter
\def\fps@figure{htbp}
\makeatother


\title{Coding 1: Lecture 1}
\author{Marc Kaufmann\\
Central European University}
\date{9/10/2019}

\begin{document}
\frame{\titlepage}

\hypertarget{coding-1-data-management-and-analysis-with-r}{%
\section{Coding 1: Data Management and Analysis with
R}\label{coding-1-data-management-and-analysis-with-r}}

\begin{frame}{Basic Admin}
\protect\hypertarget{basic-admin}{}

\begin{itemize}
\tightlist
\item
  Weeks 1 to 6: Mondays 13:30-15:10
\item
  Weeks 7 to 12: Mondays 17:30-19:10
\item
  Instructor: Marc Kaufmann (call me Marc)
\item
  Teaching Assistant: Júlia Hermann gives 3 sessions

  \begin{itemize}
  \tightlist
  \item
    17:30-19:10 on Tuesday October 8th
  \item
    17:30-19:10 on Tuesday November 5th
  \item
    17:30-19:10 on Tuesday November 19th
  \end{itemize}
\end{itemize}

\end{frame}

\begin{frame}{Who am I?}
\protect\hypertarget{who-am-i}{}

\pause{}

\begin{itemize}
\tightlist
\item
  Assistant Professor in Economics and Business
\item
  Research in Psychology and Economics (aka Behavioral Economics)
\end{itemize}

\pause{}

Relevant for this class:

\begin{itemize}
\tightlist
\item
  Collect my own data in (mostly online) experiments
\item
  Analyze said data: Fairly basic, since design is up to me
\item
  Program in Racket, bash/unix, Python, and R (in that order of
  competence)
\end{itemize}

\pause{}

Most importantly: I am good at getting help.

\end{frame}

\begin{frame}[fragile]{Goal for the class in one sentence}
\protect\hypertarget{goal-for-the-class-in-one-sentence}{}

\pause{}

\begin{quote}
Generate basic insights\pause{} from existing data\pause{} that is
small\pause{} and relational\pause{} in a reproducible and replicable
manner.
\end{quote}

\pause{}

Or expressed in R (pseudo-)code:

\pause{}

\begin{Shaded}
\begin{Highlighting}[]
\KeywordTok{library}\NormalTok{(msc_ba)}

\NormalTok{eureka_or_bust <-}\StringTok{ }\NormalTok{your_great_ideas }\OperatorTok\StringTok{ }
\StringTok{  }\CommentTok{# We need data to figure this out. Let's...}
\StringTok{  }\KeywordTok{collect_data}\NormalTok{() }\OperatorTok
\StringTok{  }\CommentTok{# Apply knowledge from this class...}
\StringTok{  }\KeywordTok{code_1}\NormalTok{()}
\end{Highlighting}
\end{Shaded}

\end{frame}

\begin{frame}[fragile]{Goals of the Course}
\protect\hypertarget{goals-of-the-course}{}

\begin{Shaded}
\begin{Highlighting}[]
\NormalTok{code_}\DecValTok{1}\NormalTok{ <-}\StringTok{ }\ControlFlowTok{function}\NormalTok{(collected_data) \{ }
\NormalTok{  collected_data }\OperatorTok\StringTok{ }
\StringTok{    }\KeywordTok{read_in}\NormalTok{() }\OperatorTok
\StringTok{    }\KeywordTok{explore}\NormalTok{() }\OperatorTok
\StringTok{    }\KeywordTok{visualize}\NormalTok{() }\OperatorTok
\StringTok{    }\KeywordTok{summarize}\NormalTok{() }\OperatorTok
\StringTok{    }\KeywordTok{clean}\NormalTok{() }\OperatorTok
\StringTok{    }\KeywordTok{tidy}\NormalTok{() }\OperatorTok
\StringTok{    }\KeywordTok{analyze}\NormalTok{() }\OperatorTok
\StringTok{    }\KeywordTok{knit}\NormalTok{()}
\NormalTok{\}}
\end{Highlighting}
\end{Shaded}

\end{frame}

\begin{frame}{Focus of the Class}
\protect\hypertarget{focus-of-the-class}{}

Since we cover much of the data analysis cycle and have little time:

\begin{itemize}
\tightlist
\item
  Focus on few libraries and commands (80/20 rule)

  \begin{itemize}
  \tightlist
  \item
    Tidyverse only: coherent set of tools with sane interface
  \end{itemize}
\item
  Focus on correct code; then maintainable; then fast
\item
  Focus on \emph{fluency}
\item
  Focus on teaching you how to learn

  \begin{itemize}
  \tightlist
  \item
    Clean code; documenting; debugging; communicating
  \item
    How to get help
  \end{itemize}
\end{itemize}

\end{frame}

\begin{frame}{Focus of the Class}
\protect\hypertarget{focus-of-the-class-1}{}

\begin{quote}
Strive to write code that is correct; maintainable; and fast. The
ordering of these adjectives is critical: correct is more important than
maintainable; maintainable is more important than fast; and fast is
important to include, because nobody wants to live with slow programs.

From ``How to Program Racket: a Style Guide'', Felleisen et al
\end{quote}

\end{frame}

\begin{frame}{Assignments and Grading}
\protect\hypertarget{assignments-and-grading}{}

\begin{itemize}
\tightlist
\item
  Participation (30\%): helping yourself and helping others

  \begin{itemize}
  \tightlist
  \item
    Includes attendance. Starting week 2, will take attendance via
    \url{https://www.youhere.org/}

    \begin{itemize}
    \tightlist
    \item
      Let me know if you cannot or do not want to use that.
    \end{itemize}
  \end{itemize}
\item
  Assignments (77-78\%): weekly assignments in the form of R Markdown
  notebooks, and assessing those of your peers

  \begin{itemize}
  \tightlist
  \item
    Grading is N\% in week N (\(N = 1, ..., 12\))
  \end{itemize}
\item
  No exam
\end{itemize}

\pause{}

Total: 107-108\%

\end{frame}

\begin{frame}{Useful Resources}
\protect\hypertarget{useful-resources}{}

When you get stuck, the following fantastic books may help:

\begin{itemize}
\tightlist
\item
  Kieran Healy's book on \emph{Data Visualization}
  (\url{http://socviz.co/})
\item
  Grolemund and Wickham's book \emph{R for Data Science}
  (\url{https://r4ds.had.co.nz/})
\end{itemize}

Additionally:

\begin{itemize}
\tightlist
\item
  Tuesdays 17:30-19:10, N13 309: Coding practice with R
\item
  Júlia's sessions
\item
  Ask questions on \url{https://discourse.trichotomy.xyz}
\end{itemize}

\end{frame}

\begin{frame}{Where you should be}
\protect\hypertarget{where-you-should-be}{}

You should:

\begin{itemize}
\tightlist
\item
  Have RStudio set up up and working
\item
  Have git set up and working
\item
  Have cloned the repository for the course
\item
  Have an account on \url{https://discourse.trichotomy.xyz}
\end{itemize}

\end{frame}

\begin{frame}{First time I teach R}
\protect\hypertarget{first-time-i-teach-r}{}

I tend to experiment quite a bit:

\begin{itemize}
\tightlist
\item
  Early on, still worth figuring out what works
\item
  Especially assignments and how to incentivize teamwork
\item
  After week 3, should be ironed out
\end{itemize}

\pause{}

Any questions?

\end{frame}

\end{document}
